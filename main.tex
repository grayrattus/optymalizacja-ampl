\documentclass{article}

\title{Wspomaganie decyzji - Zad. 1}
\author{Artur Dziedziczak}
\date{\today}

\usepackage{microtype}
\usepackage[
    backend=biber,
    natbib=true,
    url=true,
    doi=true,
    eprint=false
]{biblatex}

\addbibresource{sample.bib}
\usepackage{gensymb}
\usepackage{graphicx}

\usepackage{hyperref}
\hypersetup{
    colorlinks=true,
    linkcolor=blue,
    filecolor=magenta,
    urlcolor=cyan,
    pdftitle={Overleaf Example},
    pdfpagemode=FullScreen,
    }

\usepackage{float}

\usepackage[utf8]{inputenc}
\usepackage{amsthm}
\usepackage[english, polish]{babel}
\usepackage[T1]{fontenc}
\usepackage{theorem}
\usepackage{listings}

\lstset{frame=tb,
  language=Bash,
  aboveskip=3mm,
  belowskip=3mm,
  showstringspaces=false,
  columns=flexible,
  basicstyle={\small\ttfamily},
  numbers=none,
  numberstyle=\tiny\color{gray},
  keywordstyle=\color{blue},
  commentstyle=\color{dkgreen},
  stringstyle=\color{mauve},
  breaklines=true,
  breakatwhitespace=true,
  tabsize=3
}

\usepackage[justification=centering]{caption}
\begin{document}

\maketitle

\section{Zadanie projektu}

\section{Analityczne sformułowanie modelu. Specyfikacje problemu decyzyjnego z do określeniem wszystkich elementów. Określenie zmiennych decyzyjnych, ograniczeń i funkcji celu.}

Formułowanie problemu zaczynam od stworzenia modelu matematycznego.
Po przeczytaniu problemu nasunęły mi się następujące spostrzeżenia:

\begin{itemize}
	\item Ograniczenia, które definiuje problem są liniowe.
	\item Ceny ciągników można zamodelować przedziałami metodą przyrostów.
	\item Nie ma potrzeby używania zmiennych binarnych ponieważ funkcja opisująca przychód dla ciągników jest wklęsła oraz wykonywana jest operacja maksymalizacji przychodu.
\end{itemize}


Identyfikacja zbiorów: 

${ S, M ,C }$ - towary

Parametry modelu:

$c_i$ - cena surowca (i=S,M,C)

$c_C_k$ - cena ciągnika $C$ dla $k$ pierwszych ciągników

$c_C_1$ - wielkość produkcji ciągników po koszcie 4000 zł

$c_C_2$ - wielkość produkcji ciągników po koszcie 3500 zł

$c_C_3$ - wielkość produkcji ciągników po koszcie 3000 zł

$c_C_4$ - wielkość produkcji ciągników po koszcie 2700 zł

$c_C - 4000 * c_c_1 + 3500 * c_c_2 + 3000 * c_c_3 + 2700 * c_c_4$ Suma przedziałów

Zmienne decyzyjne:

$w_i$ - liczba wyprodukowanego surowca (i= S,M,C)

$s_i$ - liczba sprzedanego surowca (i= S,M,C)

$t_i$ - liczba surowców wyprodukowanych do stworzenia innych rzeczy (i= S,M,C)

$p$ - liczba dostępnych zasobów pracy

Funkcja celu:

$z <= 900 * s_S + 2500 * s_M + (4000 * c_C_1 + 3500 * c_C_2 + 3000 * c_C_3 + 2700 * c_C_4) - kp$

gdzie:

$z$ - zysk, który maksymalizuję

$kp$ - koszt produkcji wyrażony przez $300 * w_S + 150 * w_M + 500 * w_C$

Ograniczenia:

Dla stali $S$

\begin{itemize}
  \item $w_S = s_S + t_S$
  \item $w_S <= 400000$
  \item $t_S <= 0.75 * w_M + w_C$
\end{itemize}

Dla $M$

\begin{itemize}
  \item $w_M = s_M + t_M$
  \item $w_M <= 50000$
  \item $t_M <= 0.5 * w_S + 0.1 w_C$
\end{itemize}

Dla $C$

\begin{itemize}
  \item $w_C = s_C + t_C$
  \item $w_C <= 550000$
  \item $t_C <= 0.5 * w_S + 5 * w_M + 3 * w_C$
\end{itemize}


Ograniczenia związane ze sprzedażą ciągników.

\begin{itemize}
  \item $s_C = c_C_1 + c_C_2 + c_C_3 + c_C_4$
  \item $0 <= c_C_1 <= 50 000$
  \item $0 <= c_C_2 <= 25 000$
  \item $0 <= c_C_3 <= 75 000$
\end{itemize}

Dla $c_C_4$ nie ma ograniczenia pomiędzy ponieważ ostatnim określonym jest 75 000.


\subsection{rozwiązanie AMPL}

\lstset{language=AMPL}
\begin{lstlisting}[caption={Model napisany w języku AMPL},label=DescriptiveLabel]
# Model

set TOWARY;
set cCIndex;

var c {TOWARY} >= 0; #cena
var cC {cCIndex}  >= 0; # cena ciągnika dla wielkości produkcji
var w {TOWARY} >= 0;
var s {TOWARY} >= 0;
var t {TOWARY} >= 0;

var p >= 0;

var koszt_prod = 300 * w['S'] + 150 * w['M'] + 500 * w['C'];

# Funkcja celu
maximize zysk: 900 * s['S'] + 2500 * s['M'] + (4000 * cC[1] + 3500 * cC[2] + 3000 * cC[3] + 2700 * cC[4]) - koszt_prod;

# ograniczenia
subject to limit_sC: s['C'] = sum {j in cCIndex} cC[j];

subject to limit_sC1: 0 <= cC[1] <= 50000;
subject to limit_sC2: 0 <= cC[2] <= 25000;
subject to limit_sC3: 0 <= cC[3] <= 75000;

subject to limitS1: w['S'] = s['S'] + t['S'];
subject to limitS2: w['S'] <= 400000;
subject to limitS3: t['S'] = 0.75 * w['M'] + w['C'];

subject to limitM1: w['M'] = s['M'] + t['M'];
subject to limitM2: w['M'] <= 50000;
subject to limitM3: t['M'] = 0.05 * w['S'] + 0.1 * w['C'];

subject to limitC1: w['C'] = s['C'] + t['C'];
subject to limitC2: w['C'] <= 550000;
subject to limitC3: t['C'] = 0.08 * w['S'] + 0.12 * w['M'];

subject to pLimit1: p = 0.5 * w['S'] + 5 * w['M'] + 3 * w['C'];
subject to pLimit2: p <= 600000;

data; # Dane

set TOWARY := C, M, S;

set cCIndex := 1,2,3,4;
\end{lstlisting}

\lstset{language=BASH}
\begin{lstlisting}[caption={Komendy uruchamiające solver na stronie https://ampl.com/cgi-bin/ampl/amplcgi},label=DescriptiveLabel]
solve;
display _varname, _var;
\end{lstlisting}

\lstset{language=BASH}
\begin{lstlisting}[caption={Wynik solwera},label=DescriptiveLabel]
MINOS 5.51: optimal solution found.
5 iterations, objective 313889221.6
:     _varname       _var        :=
1    'cC[1]'          50000
2    'cC[2]'          25000
3    'cC[3]'          48377.2
4    'cC[4]'              0
5    "w['C']"        138323
6    "w['M']"         21556.9
7    "w['S']"        154491
8    "s['C']"        123377
9    "s['M']"             0
10   "s['S']"             0
11   "t['C']"         14946.1
12   "t['M']"         21556.9
13   "t['S']"        154491
14   p                6e+05
15   koszt_prod   118743000
;
\end{lstlisting}

\end{document}
