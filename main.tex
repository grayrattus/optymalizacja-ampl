\documentclass{article}

\title{Wspomaganie decyzji - Zad. 1}
\author{Artur Dziedziczak}
\date{\today}

\usepackage{microtype}
\usepackage[
    backend=biber,
    natbib=true,
    url=true,
    doi=true,
    eprint=false
]{biblatex}

\addbibresource{sample.bib}
\usepackage{gensymb}
\usepackage{graphicx}

\usepackage{hyperref}
\hypersetup{
    colorlinks=true,
    linkcolor=blue,
    filecolor=magenta,
    urlcolor=cyan,
    pdftitle={Overleaf Example},
    pdfpagemode=FullScreen,
    }

\usepackage{float}

\usepackage[utf8]{inputenc}
\usepackage{amsthm}
\usepackage[english, polish]{babel}
\usepackage[T1]{fontenc}
\usepackage{theorem}
\usepackage{listings}

\lstset{frame=tb,
  language=Bash,
  aboveskip=3mm,
  belowskip=3mm,
  showstringspaces=false,
  columns=flexible,
  basicstyle={\small\ttfamily},
  numbers=none,
  numberstyle=\tiny\color{gray},
  keywordstyle=\color{blue},
  commentstyle=\color{dkgreen},
  stringstyle=\color{mauve},
  breaklines=true,
  breakatwhitespace=true,
  tabsize=3
}

\usepackage[justification=centering]{caption}
\begin{document}

\maketitle

\section{Instrukcja instalacji i uruchomienia aplikacji w Node.js}



\section{Zadanie projektu}

\section{Analityczne sformułowanie modelu. Specyfikacje problemu decyzyjnego z do określeniem wszystkich elementów. Określenie zmiennych decyzyjnych, ograniczeń i funkcji celu.}

Formułowanie problemu zaczynam od stworzenia modelu matematycznego.
Po przeczytaniu problemu nasunęły mi się następujące spostrzeżenia:

\begin{itemize}
	\item Ograniczenia, które definiuje problem są liniowe.
	\item Ceny ciągników można zamodelować przedziałami metodą przyrostów.
	\item Nie ma potrzeby używania zmiennych binarnych ponieważ funkcja opisująca przychód dla ciągników jest wklęsła oraz wykonywana jest operacja maksymalizacji przychodu.
\end{itemize}


Identyfikacja zbiorów: 

${ S, M ,C }$ - towary

Parametry modelu:

$c_i$ - cena surowca (i=S,M,C)

$c_i_k$ - cena ciągnika $C$ dla $k$ pierwszych ciągników

$c_C_1$ - wielkość produkcji ciągników po koszcie 4000 zł

$c_C_2$ - wielkość produkcji ciągników po koszcie 3500 zł

$c_C_3$ - wielkość produkcji ciągników po koszcie 3000 zł

$c_C_4$ - wielkość produkcji ciągników po koszcie 2700 zł

$c_C - 4000 * c_c_1 + 3500 * c_c_2 + 3000 * c_c_3 + 2700 * c_c_4$ Suma przedziałów

Zmienne decyzyjne:

$w_i$ - liczba wyprodukowanego surowca (i= S,M,C)

$s_i$ - liczba sprzedanego surowca (i= S,M,C)

$t_i$ - liczba surowców wyprodukowanych do stworzenia innych rzeczy (i= S,M,C)

$p$ - liczba dostępnych zasobów pracy

Funkcja celu:

$z <= 900 * s_S + 2500 * s_M + (4000 * c_C_1 + 3500 * c_C_2 + 3000 * c_C_3 + 2700 * c_C_4) - kp$

gdzie:

$z$ - zysk, który maksymalizuję

$kp$ - koszt produkcji wyrażony przez $300 * w_S + 150 * w_M + 500 * w_C$









\lstset{language=AMPL}
\begin{lstlisting}
\end{lstlisting}

\end{document}
