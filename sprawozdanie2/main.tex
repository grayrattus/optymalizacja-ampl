\documentclass{article}

\title{Wspomaganie decyzji - Zad. 1}
\author{Artur Dziedziczak}
\date{\today}

\usepackage[table]{xcolor}% http://ctan.org/pkg/xcolor
\usepackage{subcaption}
\usepackage{microtype}
\usepackage[
    backend=biber,
    natbib=true,
    url=true,
    doi=true,
    eprint=false
]{biblatex}
\usepackage[a4paper, total={6in, 8in}]{geometry}

\addbibresource{sample.bib}
\usepackage{gensymb}
\usepackage{graphicx}

\usepackage{hyperref}
\hypersetup{
    colorlinks=true,
    linkcolor=blue,
    filecolor=magenta,
    urlcolor=cyan,
    pdftitle={Wspomaganie decyzji - Zad. 1 Artur Dziedziczak},
    pdfpagemode=FullScreen,
    }

\usepackage{float}

\usepackage[utf8]{inputenc}
\usepackage{amsthm}
\usepackage[english, polish]{babel}
\usepackage[T1]{fontenc}
\usepackage{theorem}
\usepackage{listings}

\lstset{frame=tb,
  language=Bash,
  aboveskip=3mm,
  belowskip=3mm,
  showstringspaces=false,
  columns=flexible,
  basicstyle={\small\ttfamily},
  numbers=none,
  numberstyle=\tiny\color{gray},
  keywordstyle=\color{blue},
  commentstyle=\color{dkgreen},
  stringstyle=\color{mauve},
  breaklines=true,
  breakatwhitespace=true,
  tabsize=3
}

\usepackage[justification=centering]{caption}
\begin{document}

\maketitle

\section{Zadanie projektu}

Rozważmy następujące uproszczone zagadnienie modelowania wyboru w warunkach niepewności:

Dysponujemy areałem $1 = 100ha$, na którym można stosować cztery rózne uprawy. Zyski z poszczególnych upraw zależą od pogody. Możliwych jest 6 scenariuszy sytuacji pogodowej.

Poniższa tabela określa zyski jednostkowe kolejnych upraw (wierszy) przy poszczególnych scenariuszach (kolumnach):

\begin{table}[H]
  \begin{center}
    \begin{tabular}{ c |  c  c   c   c   c   c  }
      & S1 & S2 & S3 & S4 & S5 & S6 \\
      \hline
      $x_1$ & 292 & 179 & 114 & 247 & 426 & 259 \\
      $x_2$ & -128 & 560 & 648 & 544 & 182 & 850 \\
      $x_3$ & 420 & 187 & 366 & 249 & 322 & 159 \\
      $x_4$ & 579 & 639 & 379 & 924 & 5 & 569 \\
      \hline
    \end{tabular} 
  \end{center}
\end{table}

Rozważmy dwa typy decyzji:

\begin{itemize}
    \item o niepodzielnych wariantach, gdy cały areał musi być przeznaczony pod jedną uprawę.
    \item o podzielnych wariantach, gdy możliwy jest wiele upraw na podzielonym areale ($x_1 + x_2 + ... + x_n = a$).
\end{itemize}

Przeanalizować  wyniki różnych kryteriów wyobru w warunkach niepewności.

Raport powinien być plikiem pdf (z ewentualnymi załącznikami).


\section{Decyzje o niepodzielnych wariantach}

Kryteria zawarte w sprawozdaniu są kryteriami, które zostały omówione w wykłądzie 4.

W tej sekcji sprawozdania nie używam języka AMPL ponieważ obliczenia i podjęcie decyzji można wykonać poprzez
prostą analizę danych w tabeli.

\subsection{Kryterium średniej}

\begin{table}[H]
  \begin{center}
    \begin{tabular}{ c |  c  c   c   c   c   c  | c | c  }
      & S1 & S2 & S3 & S4 & S5 & S6 & Suma & Średnia \\
      \hline
      $x_1$ & 292 & 179 & 114 & 247 & 426 & 259 & 1517 & 253 \\
      $x_2$ & -128 & 560 & 648 & 544 & 182 & 850 & 2656 & 443 \\
      $x_3$ & 420 & 187 & 366 & 249 & 322 & 159 & 1703 & 284 \\
      $x_4$ & 579 & 639 & 379 & 924 & 5 & 569 & \cellcolor{orange!25} 3095 & \cellcolor{orange!25} 516 \\
      \hline
    \end{tabular} 
    \caption{\label{table:avg}Tabela zawiera informację o średniej i sumie poszczególnych kolumn.}
  \end{center}
\end{table}

Wnioski:

\begin{itemize}
    \item Ponieważ ilość wariantów jest taka sama decyzję można podjąć już po określeniu sumy wyników. Średnia z najwyższej sumy
      w tych warunkach zawsze będzie najwyższa.
    \item Podjęta decyzja to $x_4$.
    \item Najgorsza możliwa decyzja do podjęcia to decyzja $x_1$.
\end{itemize}

\subsection{Kryterium optymistyczne i pesymistyczne}

\begin{table}[H]
  \begin{center}
    \begin{tabular}{ c |  c  c   c   c   c   c  | c | c  }
      & S1 & S2 & S3 & S4 & S5 & S6 & Maksimum & Minimum \\
      \hline
      $x_1$ & 292 & 179 & 114 & 247 & 426 & 259 & 426 & 114 \\
      $x_2$ & -128 & 560 & 648 & 544 & 182 & 850 & 850 & -128 \\
      $x_3$ & 420 & 187 & 366 & 249 & 322 & 159 & 420 & \cellcolor{red!25} 159 \\
      $x_4$ & 579 & 639 & 379 & 924 & 5 & 569 & \cellcolor{green!25} 924 & 5 \\
      \hline
    \end{tabular} 
    \caption{\label{table:optpes}Tabela zawiera informację o maksymalnym zysku i minimalnej stracie/zysku.}
  \end{center}
\end{table}

Wnioski:

\begin{itemize}
  \item Kryterium optymistyczne wybiera największy dostępny zyzk (zielony).
  \item Kryterium optymistyczne wybiera najmniejszą możliwą stratę lub w tym przypadku zysk (czerwony).
  \item Tak jak w przypadku kryterium średniej ~\ref{table:avg} wybór kryterium optymistycznego jest identyczny $x_4$.
\end{itemize}

\subsection{Kryterium Hurwicza}

\begin{table}[H]
  \begin{center}
    \begin{tabular}{ c | c c c c c c c c c c c }
      \alpha & 0 & 0.1 & 0.2 & 0.3 & 0.4 & 0.5 & 0.6 & 0.7 & 0.8 & 0.9 & 1 \\
      \hline
      $x_1$ & 114 & 145 & 176 & 208 & 239 & 270 & 301 & 332 & 364 & 395 & 426 \\
      $x_2$ & -128 &-30 & 68 & 165 & 263 & 361 & 459 & 557 & 654 & 752 & 850 \\
      $x_3$ & \cellcolor{red!25} 159 & \cellcolor{red!25} 185 & \cellcolor{red!25}211 & 237 & 263 & 290 & 316 & 342 & 368 & 394 & 420 \\
      $x_4$ & 5 & 97 & 189 & \cellcolor{green!25}281 & \cellcolor{green!25}373 & \cellcolor{green!25}464 & \cellcolor{green!25}556 & \cellcolor{green!25}648 & \cellcolor{green!25}740 & \cellcolor{green!25}832 & \cellcolor{green!25}924 \\
      \hline
    \end{tabular} 
    \caption{\label{table:optpes} Tabela zawiera wartości wartości funkcji H oraz kombinację }
  \end{center}
\end{table}

Wnioski:

\begin{itemize}
  \item 
\end{itemize}

\subsection{Minimalizacja maksymalnego żalu}

\begin{table}[H]
  \begin{center}
    \begin{tabular}{ c |  c  c   c   c   c   c  }
      & S1 & S2 & S3 & S4 & S5 & S6 \\
      \hline
      $x_1$ & 292 & 179 & 114 & 247 & 426 & 259 \\
      $x_2$ & -128 & 560 & 648 & 544 & 182 & 850 \\
      $x_3$ & 420 & 187 & 366 & 249 & 322 & 159 \\
      $x_4$ & 579 & 639 & 379 & 924 & 5 & 569 \\
      \hline
      maks. & 579 & 639 & 648 & 924 & 426 & 850 \\
    \end{tabular} 
    \caption{\label{table:areals} Tabela zawiera maksymalny zysk dla poszczególnych areałów.}
  \end{center}
\end{table}

\begin{table}[H]
  \begin{center}
    \begin{tabular}{ c |  c  c   c   c   c   c | c  }
      & S1 & S2 & S3 & S4 & S5 & S6 & Maksimum \\
      \hline
      $x_1$ & 287 & 460 & 534 & 677 & 0   & 591 & 677 \\
      $x_2$ & 707 & 79  & 0   & 380 & 244 & 0   & 707 \\
      $x_3$ & 159 & 452 & 282 & 675 & 104 & 691 & 691 \\
      $x_4$ & 0   & 0   & 269 & 0   & 421 & 281 & \cellcolor{green!25}421 \\
      \hline
    \end{tabular} 
    \caption{\label{table:zal} Wyniki obliczeń dla minimalizacji masymalnego żalu. }
  \end{center}
\end{table}

Wnioski:

\begin{itemize}
  \item 
\end{itemize}


\section{Decyzje o podzielnych wariantach}
\end{document}
